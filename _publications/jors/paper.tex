%% Journal of Open Research Software Latex template -- Created By Stephen Bonner and John Brennan, Durham Universtiy, UK.

\documentclass{jors}

%% Set the header information
\pagestyle{fancy}
\definecolor{mygray}{gray}{0.6}
\renewcommand\headrule{}
\rhead{\footnotesize 3}
\rhead{\textcolor{gray}{UP JORS software Latex paper template version 0.1}}

\usepackage{biblatex}
\usepackage{hyperref}
\bibliography{paper}


\begin{document}

%% {\bf Software paper for submission to the Journal of Open Research Software} \\

%% To complete this template, please replace the blue text with your own. The paper has three main sections: (1) Overview; (2) Availability; (3) Reuse potential. \\

%% Please submit the completed paper to: editor.jors@ubiquitypress.com

%% \rule{\textwidth}{1pt}

\section*{(1) Overview}

\vspace{0.5cm}

\section*{Title}

PFHub: The Phase-Field Community Hub

\section*{Paper Authors}

1. Wheeler, Daniel; (corresponding author) 1\\
2. Keller, Trevor; 1\\
3. DeWitt, Stephen J.; 6\\
4. Jokisaari, Andrea M.; 5\\
5. Schwen, Daniel; 5\\
6. Guyer, Jonathan E.; 1\\
7. Aagesen, Larry; 5\\
8. Heinonen, Olle G.; 2\\
9. Voorhees, Peter W.; 4\\
10. Warren, James A.; 3

\section*{Paper Author Roles and Affiliations}

1. Materials Science and Engineering Division, \\
Material Measurement Laboratory, \\
National Institute of Standards and Technology,\\
Gaithersburg, MD 20899 USA

2. Argonne National Laboratory, \\
Lemont, IL 60439 USA

3. Laboratory Office, \\
Material Measurement Laboratory, \\
National Institute of Standards and Technology, \\
Gaithersburg, MD 20899 USA

4. Department of Materials Science and Engineering, \\
Northwestern University, \\
Evanston, IL 60208 USA

5. Fuel Modeling and Simulation Department, \\
Idaho National Laboratory, \\
Idaho Falls, ID 83415 USA

6. Materials Science and Engineering Department, \\
University of Michigan, \\
Ann Arbor, MI 48109 USA

7. Materials Science and Engineering Department, \\
University of Michigan, \\
Ann Arbor, MI 48109 USA

\section*{Abstract}

Scientific research communities often require an online portal to
summarize a shared challenge, collect attempts at a solution, and
present a quantitative comparison of past attempts in a compelling
way. An examplar of such a portal is $\mu$MAG~\cite{mumag}. The
reusable PFHub framework leverages existing online services to build a
static portal website that is considerably easier to deploy and
maintain without sacrificing content or scope. The first deployment of
the PFHub framework supports phase-field practitioners and code
developers participating in an effort to improve quality assurance for
phase-field codes.

\section*{Keywords}

\textcolor{blue}{keyword 1; keyword 2; etc. \\
Keywords should make it easy to identify who and what the software will be useful for.}

\section*{Introduction}

\textcolor{blue}{An overview of the software, how it was produced, and the research for which it has been used, including references to relevant research articles. A short comparison with software which implements similar functionality should be included in this section. }

\section*{Implementation and architecture}

\textcolor{blue}{How the software was implemented, with details of the architecture where relevant. Use of relevant diagrams is appropriate. Please also describe any variants and associated implementation differences.}


\section*{Quality control}

\textcolor{blue}{Detail the level of testing that has been carried out on the code (e.g. unit, functional, load etc.), and in which environments. If not already included in the software documentation, provide details of how a user could quickly understand if the software is working (e.g. providing examples of running the software with sample input and output data). }

\section*{(2) Availability}
\vspace{0.5cm}
\section*{Operating system}

\textcolor{blue}{Please include minimum version compatibility.}

\section*{Programming language}

\textcolor{blue}{Please include minimum version compatibility.}

\section*{Additional system requirements}

\textcolor{blue}{E.g. memory, disk space, processor, input devices, output devices.}

\section*{Dependencies}

\textcolor{blue}{E.g. libraries, frameworks, incl. minimum version compatibility.}

\section*{List of contributors}

\textcolor{blue}{Please list anyone who helped to create the software (who may also not be an author of this paper), including their roles and affiliations.}

\section*{Software location:}

{\bf Archive} \textcolor{blue}{(e.g. institutional repository, general repository) (required – please see instructions on journal website for depositing archive copy of software in a suitable repository)}

\begin{description}[noitemsep,topsep=0pt]
	\item[Name:] \textcolor{blue}{The name of the archive.}
	\item[Persistent identifier:] \textcolor{blue}{e.g. DOI, handle, PURL, etc.}
	\item[Licence:] \textcolor{blue}{Open license under which the software is licensed.}
	\item[Publisher:]  \textcolor{blue}{Name of the person who deposited the software.}
	\item[Version published:] \textcolor{blue}{The version number of the software archived.}
	\item[Date published:] \textcolor{blue}{dd/mm/yy}
\end{description}



{\bf Code repository} \textcolor{blue}{(e.g. SourceForge, GitHub etc.) (required)}

\begin{description}[noitemsep,topsep=0pt]
	\item[Name:] \textcolor{blue}{The name of the archive.}
	\item[Persistent identifier:] \textcolor{blue}{e.g. DOI, handle, PURL, etc.}
	\item[Licence:] \textcolor{blue}{Open license under which the software is licensed.}
	\item[Date published:] \textcolor{blue}{dd/mm/yy}
\end{description}

{\bf Emulation environment} \textcolor{blue}{(if appropriate)}

\begin{description}[noitemsep,topsep=0pt]
	\item[Name:] \textcolor{blue}{The name of the archive.}
	\item[Persistent identifier:] \textcolor{blue}{e.g. DOI, handle, PURL, etc.}
	\item[Licence:] \textcolor{blue}{Open license under which the software is licensed.}
	\item[Date published:] \textcolor{blue}{dd/mm/yy}
\end{description}

\section*{Language}

\textcolor{blue}{Language of repository, software and supporting files.}

\section*{(3) Reuse potential}

\textcolor{blue}{Please describe in as much detail as possible the ways in which the software could be reused by other researchers both within and outside of your field. This should include the use cases for the software, and also details of how the software might be modified or extended (including how contributors should contact you) if appropriate. Also you must include details of what support mechanisms are in place for this software (even if there is no support).}

\section*{Acknowledgements}

\textcolor{blue}{Please add any relevant acknowledgements to anyone else who supported the project in which the software was created, but did not work directly on the software itself.}

\section*{Funding statement}

\textcolor{blue}{If the software resulted from funded research please give the funder and grant number.}

\section*{Competing interests}

\textcolor{blue}{If any of the authors have any competing interests then these must be declared. The authors’ initials should be used to denote differing competing interests. For example: “BH has minority shares in [company name], which part funded the research grant for this project. All other authors have no competing interests." \\
If there are no competing interests, please add the statement:
“The authors declare that they have no competing interests.” }

%% \section*{References}

\printbibliography

%% \textcolor{blue}{Please enter references in the Harvard style and include a DOI where available, citing them in the text with a number in square brackets, e.g. \\ }

%% \textcolor{blue}{[1] Piwowar, H A 2011 Who Shares? Who Doesn't? Factors Associated with Openly Archiving Raw Research Data. PLoS ONE 6(7): e18657. DOI: \\ http://dx.doi.org/10.1371/journal.pone.0018657.}

\vspace{2cm}

\rule{\textwidth}{1pt}

{ \bf Copyright Notice} \\
Authors who publish with this journal agree to the following terms: \\

Authors retain copyright and grant the journal right of first publication with the work simultaneously licensed under a  \href{http://creativecommons.org/licenses/by/3.0/}{Creative Commons Attribution License} that allows others to share the work with an acknowledgement of the work's authorship and initial publication in this journal. \\

Authors are able to enter into separate, additional contractual arrangements for the non-exclusive distribution of the journal's published version of the work (e.g., post it to an institutional repository or publish it in a book), with an acknowledgement of its initial publication in this journal. \\

By submitting this paper you agree to the terms of this Copyright Notice, which will apply to this submission if and when it is published by this journal.


\end{document}
